%-------------------------------------------------------------------------------
\section{Conclusion}
%-------------------------------------------------------------------------------


Serverless was and is a great option for developers whose load varies and
providers who don't want to keep resources idle for processes that reserved them
and might need them soon. However, the reality of serverless today does not
match up in many different ways; most notably it has a variance in latencies
that functions experience that is not tolerable for latency sensitive workloads.
The common approach to this is to work on cold starts. 

This paper asks what comes next; once cold starts in the single digit ms range,
which we are starting to see in research schedulers, are commercially available,
have we figured out a system that can run serverless in its ideal form?

We show that this is not the case, and that once more latency senstive functions
are able to run alongside the usual map reduce and image resize functions, we
will need some way of prioritising which functions are latency sensitive in
order to keep their overall latencies acceptable.

We propose a new scheduler, \sys{}, that introduces \emph{\priceclass{}es}.
Developers assign each function they want to run to a \priceclass{}, which is a
priority that \sys{} then enforces at invocation, both in placing the function
and on the machine level by running priority scheduling. We show that \sys{} is
able to enforce priorities and keep high \class{} functions latencies stable
even under high load.