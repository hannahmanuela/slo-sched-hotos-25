%-------------------------------------------------------------------------------
\section{Introduction}
\label{s:intro}
%-------------------------------------------------------------------------------

A world where cloud compute runs as serverless functions is attractive to
developers and providers: developers pay only for what they use, while having
access to many resources when needed; and cloud providers can maximize
utilization by intelligent scheduling of functions, in contrast to systems in
which clients reserve resources which then often lie idle.

Web applications are an example of a workload that could potentially benefit
from serverless, due to their bursty load patterns. However, they rarely use use
serverless~\cite{reddit-serverless1, reddit-serverless2, not-lambda-blog}. One
reason is that even short functions sometimes suffer long delays under
serverless: in a small benchmark on AWS (described in~\autoref{s:motivation}), we found that total execution times for a
simple hello world function that sleeps for 20 ms ranged from 20 to 400ms. Small
response time differences can have a large negative impact in interactive
applications~\cite{amz-page-load,google-page-load}; the maximum acceptable
latency for a user-facing function is closer to 100ms~\cite{page-load-time}.

While a well-known cause of variable latency is cold start, research is putting
cold start times of a few milliseconds within reach~\cite{sigmaos,mitosis}. If
cold start is no longer an obstacle, will serverless be ready to support web
applications?

A major remaining problem for latency-sensitive tasks on serverless is the
\emph{\problem{}}: under high total load, different tenants' functions compete
for resources, and thus can drive up each other's delays. Whether the outcome is
acceptable depends on how the provider schedules different tenants' functions,
as well as the functions' latency senstivity.

This paper proposes \sys{}, a serverless scheduler that addresses the \problem{}
by ensuring that latency sensitive functions aren't blocked behind background
work.

Designing \sys{} faces multiple challenges. One of them is that the scheduler
needs a basis on which to compare the importances of different tenants'
functions. Each individual tenant can be expected to understand which of its own
functions are most in need of low latency, but how to compare these decisions
across tenants? To achieve this global comparability, tenants declare a
\emph{\priceclass{}} for each function invocation. The \priceclass{} is the
amount of money the tenant will pay per unit of CPU time. The scheduler can then
make scheduling decisions among different tenants' functions by comparing their
price classes.
% It also incentivizes usage of the lower \class{}es for functions that are less
% latency sensitive, since they can be run cheaply.

Another challenge is rapid placement of functions in a large cluster of servers.
Tracking idle or pre-emptible resources is difficult when both the number of new
function invocations and the amount of resources are large.

A third key challenge in designing \sys{} is memory management. The provider
faces a tension: high CPU utilization requires packing many functions onto each
machine, but not so many that the machine runs out of memory. Current systems
avoid memory exhaustion by requiring developers to declare the maximum amount of
memory each function will use. However, memory use is difficult to predict and
varies across invocations. Instead, \sys{} charges developers based on the
amount of memory actually used, and requires no bound to be set.~\Sys{} thus
faces the challenging proposition of blindly placing functions not knowing how
much memory they will use, but still needing CPU utilization to be high.
