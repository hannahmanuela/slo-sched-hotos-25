%-------------------------------------------------------------------------------
\section{Introduction}
%-------------------------------------------------------------------------------

A world where cloud compute is run in the format of serverless jobs is
attractive to developers and providers: developers pay only for what they use,
while having access to many resources when needed; and cloud providers have
control over scheduling and can use that to drive up utilization, rather than
needing to hold idle resources available for clients who reserved them.


However, there remain realities that make serverless today fundamentally
infeasible for workloads that in theory are a good fit --- for instance web
servers, which are often bursty and have inconsistent load, but are rarely run
completely in serverless offerings~\cite{reddit-serverless1}, such as AWS lambda.
One of the things that makes running a web server on lambda infeasible is lambda
invocations' variable end to end latencies: in a small benchmark (broken down in
Section~\ref{motivation}) we found that total execution time latencies for a
simple hello world function that sleeps for 20 ms ranged from 20 to 400ms.

An obvious and well-known problem that causes these high and variable latencies
is that of cold starts. However, this paper takes the position that we are well
underway to reaching low single digit ms cold start times, with current
state-of-the-art research systems pushing into single digit
territory~\cite{sigmaos,mitosis}. Which begs the question: if cold start is fast enought
that more latency sensitive applications, like web servers, can have a cold
start on the critical path, are we then done? Will serverless then be, at least
from an infrastructure perspective, ready to support these sorts of workloads?

This paper argues that no, there still remains a fundamental obstacle to running
such a latency sensitive workload on serverless: queueing and delay within the
system. We cannot expect that the load will always fit in the resources
providers have, and so some work has to be queued or otherwise degraded. In the
world of long running servers, developers avoid degradation of access to
resources by giving latency critical services reservations; but reserving
resources in serverless doesn't work. 

% In a serverless setting as they exist today, running at high utilization
% almost necessarily will mean there is queueing; in fact in
% Section~\ref{motivation} we show that is already the case in AWS lambda.

What developers care about in the end is that the jobs that are important run
quickly. The problem we address is that, in a world where cold starts are fast
and latency sensitive work is able running alongside map reduce and image
processing jobs, important jobs might end up behind unimportant ones, waiting to
be placed on machines and to get access to resources once placed. To address
this problem, we propose \sys{}, a scheduling system that has
\textit{\priceclass{}es} as the central metric associated with each job.
Priority in the system is directly paid for through \priceclass{}es, and all of
the resource allocation decisions in \sys{} are made on the basis of
\priceclass{}.

\Sys{} has multiple goals it needs to achieve and challenges it needs to
address.

One key goal is that \sys{} needs to be able to support a multi tenant
environment. This is enabled by the \priceclass{}es being a universal metric.
Rather than dealing in a relative ordering of developers' functions by
importance, which would be difficult to compare across developers, the
connection to money allows the \class{} to have meaning in an absolute as well
as a relative way. It also incentivizes usage of the lower \class{}es for jobs
that are less important, since they can be executed much more cheaply.


Another important goal is that of placing jobs quickly enough. For example, a
job that takes 20ms to run cannot spend 50ms in scheduler queues and waiting for
an execution environment before even starting to run. Even when they're short,
cold starts will take up a significant portion of the time a job can handle
waiting to run. So the challenge is knowing where the free and idle resources
are, or finding out quickly, in a setting where both the number of new jobs
coming in and the amount of resources are extremely large.


Finally, a key challenge in designing \sys{} is that of managing memory. For
compute, cores can always be timshared or processes preempted, but the buck
stops once a machine is out of memory. This challenge is made more difficult by
the fact that \sys{} does not require developers to give memory usage limits for
their jobs. This serves the purpose of extending the serverless on-demand
structure to include memory, and moves away from the usual bin packing with
overprovisioning problem.~\Sys{} is thus faced with the challenging proposition
of blindly placing jobs not knowing how much memory they will use, but still
needing memory utilization to be high.
 
