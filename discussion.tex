%-------------------------------------------------------------------------------
\section{Discussion}
%-------------------------------------------------------------------------------


We have seen that having \priceclass{}es helps \sys{} deal with overprovisioning
for compute. Traditionally, managing the response time for important jobs
requires the overall load on the system to be low, because slowdown is averaged
across all the jobs that can share resources. However, having \class{}es allows
\sys{} to be targeted about how compute resources are shared: rather than
averaging out the slowdown caused by overprovisioning across all the jobs by
having them time share the cores, \sys{} can use \class{} to decide which job
has to wait. This means that in priority scheduling the amount of time a job
spends waiting for resources is only defined by the load of jobs with equal or
higher \class{}.

The flipside of this is that it is possible that the entire load of the
datacenter will be such that low \class{} jobs are starved. This is acceptable
and in fact desirable for small amounts of time, but keeping this effect in
check requires managing the overall load. 

We propose to solve this problem by ensuring that it can never be the case the
there is so much load on high \class{} jobs that the data center will be full
with them to such a degree that other jobs can't run. There is evidence for and
we expect that load on a high level will mostly be stable, with diurnal and
annual patterns.\cite{TODO}

This means that providers can mostly choose the rough breakdown of the load they
will have at any given time (ie they can choose a percentage of how many jobs
they want in each \class{}, and change it by adjusting prices or not allowing
users to select that level anymore). 



% other things we could talk about:
% i/o?
% 
