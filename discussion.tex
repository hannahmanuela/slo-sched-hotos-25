%-------------------------------------------------------------------------------
\section{Discussion}
%-------------------------------------------------------------------------------



There are still a number of open questions that the design only partially
addresses, or does not have a completely satisfactory answer to.

\textbf{Economics of memory.}
One of the key pieces of the design is the notion that jobs have an associated
amount of memory, and when the job is placed that estimate is used in order to
understand the consequences of placing a job (whether this will result in a kill
of not). However, we did not discuss the payment model: do developers only pay
for the memory they use, or do they pay based on the max amount of the memory
they give? The former would result in gross overestimations because there is no
downside to being on the cautious side, while the latter breaks the central
maxim that developers only pay for what they use.

In the world where developers only pay for what they use, asking for a limit
with no repurcussions if the estimate is off would be useless; everyone would
just put the largest amount allowed. Adding penalties for being off seems
unreasonable: some jobs simply have a high variance in their memory usage and
penalizing developers that run such jobs is undesirable. 

Ideally, memory would be a pay-per-use model and no further information would be
requied; however that would require being able to reactively deal with memory
pressure extremely quickly and well.

\textbf{Dealing with memory pressure.}
Whether the memory usage per job is capped or not, because actual memory usage
varies between job invocations, having high memory utilization will requite
overprovisioning machines on memory. The current design deals with high memory
pressure by simply killing the process that is the most convenient: lowest
priority and using enough memory to make a difference.

This is not really a satisfying solution. Ideally, the system would avoid the
situation altogether, or if it occured would be able to solve it without
killing, ie wasting resources. 

Well-know solutions for this are profiling or reactively snapshotting/paging.
Profiling is improving as ML models improve, and might be a good use case
(especially for jobs with a high variance of memory usage, using a model that is
given the inputs to invocations could work well, since the inputs are likely to
be the determining factor for memory usage). On the other hand, profiling is
still just an estimate that could always be wrong, and the system needs to be
able to handle that. If we are able to come up with a mechanism that is reactive
and at the right time scale, ie acts quickly enough that there is no buildup and
the problem goes away, that would be ideal. \\
One option for this might be snapshotting: lower priority functions that we are now
reactively killing might be allowed to snapshot themselves and then be placed
somewhere else and re-started. In this case, the timescale would have to be such
that the snapshotting does not (in its use of memory or compute) prohibively
block the other jobs on the machine from running. Another option could be
paging: the lower priority processes' memory are paged out; later when there is
lower load and they start running again their latency will be affected but they
are lower priority so we don't care as much (since developers pay per usage for
compute one could imagine some sort of recompense for the runtime that job pays
for having been paged out).


\textbf{Does priority increase with waiting.}
One of the observations the system design is built on is that Processor Sharing
(PS) is not the right approach in a serverless setting, where jobs run once and
then are done. Instead, \sys{} uses preemptive priority scheduling, where at any
moment in time the process running is the one with the highest priority on the
machine. This means that if it is possible for load in the highest priority
class to take up all the resources, nothing of any lower resource would ever run
at all. 

It is not necessarily clear that this is desirable, for instance in the web
server example if a user profile view has waited for long enough it seems fine
for a landing page view to be blocked for the time it takes while the profile
view job runs. This would point to it being desirable that processes gain
priority as they wait. On the other hand, this would not be true of a map reduce
job: page views should never be interrupted because a map review job had been
waiting for a long time.

One solution to this problem is ensure that it can never be the case the there
is so much load on high priority jobs that the data center will be full with
them to such a degree that high-ish priority jobs can't run. There is evidence
for and we expect that load will mostly be stable: this supported by the strong
law of large numbers (it is very unlikely that all the jobs' random  bursts
align), and can be seen in the ways AWS prices things: spot instances are sold
at a market rate determined by the amount of resources available in a given
zone\cite{TODO}, and the market rate is experientially very steady over
time\cite{TODO}. The prices for lambdas and ec2 instances also only changes very
slowly\cite{TODO}. This means that providers can mostly choose the rough
breakdown of the load they will have at any given time (ie they can choose a
percentage of how many jobs they want in each priority, and change it by
adjusting prices or not sllowing users to select that level anymore). 

\hmng{We discuss what a good breakdown would look like in the next section? Put eval
next? Or just don't discuss and leave it at that, although that seems a little
vague.}


