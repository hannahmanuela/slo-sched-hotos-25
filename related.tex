\section{Related Work}

Many other projects have explored how to do better serverless scheduling.
 
Some projects simply change the mechanisms and try to optimize observed behavior
by trying out different scheduling policies. Sparrow does this on the
distributed level, and takes away that a distributed scheduler using
power-of-k-choices has performance close to a centralized global scheduler.
Hermod performs a from-first-principles analysis of different scheduling
paradigms through a simulator, along three dimensions: late-vs early binding, PS
vs FCFS vs SRPT, and least loaded vs locality based vs random load balancing.
Kairos does a distributed approximation of Least Attained Service, as a way of
avoiding trying to estimate future behavior based on past performance. 

Other projects do just that: generate better scheduling decisions by getting
access to more information about the workload. There are two different ways of
going about this: learn the behaviors, or ask developers to specify. 

ALPS, for instance, observes and learns the behaviors of existing functions and
then makes scheduling decisions based on those. Another example is Morpheus,
which learns SLOs from historical runs, and then translates these to recurring
reservations. 

Other projects ask developers to specify the desired behavior. Each has come up
with a different way of expressing priorities or policies, and then built a
scheduler to implement those priorities. Sequoia, for instance, adds a metric of
QOS, and then builds a prioritization in front of AWS lambda. Another project
creates a language of Allocation Priority Policies (APP), which is a declarative
language to express policies, then builds a scheduler around that.\hmng{ should I
talk about the actual contents of the languages and how they express priorities?
Some are pretty complex, but this is also the category that we fall into, and I
talk about it for AWS so; but AWS’s is also way simpler } AWS lambda itself also
goes this route, by offering two different ways for developers to influence
lambda function scaling: provisioned and reserved concurrency. Provisioned
concurrency specifies a number of instances to keep warm, and reserved
concurrency does the same but ensures that those warm instances are kept for a
specific function. 

XX goes the route of asking developers to provide information about desired
behavior, but has a much simpler interface than Sequoia or APP. Lambda also has
a simple interface, but it is focused on mitigating effects of cold starts,
rather than actually prioritizing the scheduling of the functions.
