\section{Related Work}

There is a large literature on scheduling for data centers but none
address the \problem.  Systems like Sparrow\cite{sparrow},
Hermod\cite{hermod}, or Kairos\cite{kairos} improve performance of
scheduling in the distributed setting by trying out and using
different scheduling policies. Unlike \sys{}, they treat all functions
equally.

Like \sys{}, many projects tailor their approach to serverless. Some systems
generate information about functions themselves to help placement decisions; for
instance ALPS\cite{alps}, which observes and learns the behaviors of existing
functions and then makes scheduling decisions based on those; or
Morpheus\cite{morpheus}, which learns SLOs from historical runs, and then
translates these to recurring reservations.~\Sys{} instead obtains the \class{}es
directly from the developers and bases its decisions solely on \class{}es.

Other papers have taken the same approach of obtaining information
from the developers. Sequoia\cite{sequoia}, for instance, creates a
metric of QOS for serverless functions. Unlike \sys{} however, Sequoia
does not implement a new scheduler, but is itself a serverless
function that manages the invocation sequence of developer's function
chains by interposing on the triggers and choosing what to invoke
when. This design does not support multi-tenancy.

Allocation Priority Policies (APP)\cite{app-paper} provides a declarative
language to express policies. The APP
language allows developers to specify custom load balancing
decisions, and the scheduler uses the developers' specification to define a
mapping of function invocations to workers. \Sys{}, on the other hand, does not
ask developers to set the load balancing policy, but rather has developers give
\sys{} the information it needs to do the load balancing itself.

%% AWS offers two different ways for developers to influence their functions'
%% scaling: provisioned and reserved concurrency\cite{aws-scaling}. Provisioned
%% concurrency specifies a number of instances to keep warm for a given function,
%% and reserved concurrency ensures that a fixed amount of the possible concurrency
%% reserved for it. This interface is bad for serverless workloads for the same
%% reasons that reservation-based systems are: it requires developers to estimate
%% their future needs and pay up front, and providers to keep those potentially
%% idle resources available.

Serverless orchestration systems like Dirigent~\cite{dirigent} are orthogonal to
\sys{}: their approaches can be combined to further reduce the latency overheads
that functions face.

\fk{scheduling based on money, spot instances}

% On the serverful side of scheduling, priorities are generally expressed via a
% latency critical/best effort binary, where latency critical processes have an
% attached amount of resource reservations, and best effort processes
% don't~\cite{kubernetes-lc-be}. Serverless schedulers that sit on top of
% Kubernetes, such as OpenFaaS~\cite{openfaas} or Knative~\cite{knative}, use
% autoscaling to keep up with load, and use the same interface as kubernetes for
% developers to express resource requirements~\cite{knative-res, openfaas-res}.
% This works well for long running servers with steady amounts of load, since
% predictable load will allow developers to make good approximations of the
% resources they will need. The serverless setting \sys{} works within is
% different because both the number of invocations and the resource usage of each
% invocation is expected to vary. 
